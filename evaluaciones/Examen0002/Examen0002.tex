\documentclass[12pt,a4paper]{article}

\usepackage[textwidth=165mm,textheight=240mm, includehead, includefoot, top =1cm, bottom=1cm,nomarginpar]{geometry}
\setlength{\parindent}{0pt}
\setlength{\parskip}{6pt plus 2pt minus 1pt}
\setlength{\emergencystretch}{3em}  % prevent overfull lines

% xetex settings
% \usepackage{fontspec,xltxtra,xunicode}
% \defaultfontfeatures{Mapping=tex-text,Scale=MatchLowercase}
% \setsansfont[Mapping=tex-text,Ligatures={Common}, Numbers={Lining}]{Helvetica Neue}
% \setmainfont[Mapping=tex-text,Ligatures={Common}, Numbers={Lining}]{Times New Roman}
% \usepackage[math-style=TeX]{unicode-math}
% \setmathfont{STIX Two Math}

%\usepackage{graphicx}
% bring in stuff from pandoc

\usepackage{color}
\usepackage{fancyvrb}
\newcommand{\VerbBar}{|}
\newcommand{\VERB}{\Verb[commandchars=\\\{\}]}
\DefineVerbatimEnvironment{Highlighting}{Verbatim}{commandchars=\\\{\}}
% Add ',fontsize=\small' for more characters per line
\usepackage{framed}
\definecolor{shadecolor}{RGB}{241,243,245}
\newenvironment{Shaded}{\begin{snugshade}}{\end{snugshade}}
\newcommand{\AlertTok}[1]{\textcolor[rgb]{0.68,0.00,0.00}{#1}}
\newcommand{\AnnotationTok}[1]{\textcolor[rgb]{0.37,0.37,0.37}{#1}}
\newcommand{\AttributeTok}[1]{\textcolor[rgb]{0.40,0.45,0.13}{#1}}
\newcommand{\BaseNTok}[1]{\textcolor[rgb]{0.68,0.00,0.00}{#1}}
\newcommand{\BuiltInTok}[1]{\textcolor[rgb]{0.00,0.23,0.31}{#1}}
\newcommand{\CharTok}[1]{\textcolor[rgb]{0.13,0.47,0.30}{#1}}
\newcommand{\CommentTok}[1]{\textcolor[rgb]{0.37,0.37,0.37}{#1}}
\newcommand{\CommentVarTok}[1]{\textcolor[rgb]{0.37,0.37,0.37}{\textit{#1}}}
\newcommand{\ConstantTok}[1]{\textcolor[rgb]{0.56,0.35,0.01}{#1}}
\newcommand{\ControlFlowTok}[1]{\textcolor[rgb]{0.00,0.23,0.31}{\textbf{#1}}}
\newcommand{\DataTypeTok}[1]{\textcolor[rgb]{0.68,0.00,0.00}{#1}}
\newcommand{\DecValTok}[1]{\textcolor[rgb]{0.68,0.00,0.00}{#1}}
\newcommand{\DocumentationTok}[1]{\textcolor[rgb]{0.37,0.37,0.37}{\textit{#1}}}
\newcommand{\ErrorTok}[1]{\textcolor[rgb]{0.68,0.00,0.00}{#1}}
\newcommand{\ExtensionTok}[1]{\textcolor[rgb]{0.00,0.23,0.31}{#1}}
\newcommand{\FloatTok}[1]{\textcolor[rgb]{0.68,0.00,0.00}{#1}}
\newcommand{\FunctionTok}[1]{\textcolor[rgb]{0.28,0.35,0.67}{#1}}
\newcommand{\ImportTok}[1]{\textcolor[rgb]{0.00,0.46,0.62}{#1}}
\newcommand{\InformationTok}[1]{\textcolor[rgb]{0.37,0.37,0.37}{#1}}
\newcommand{\KeywordTok}[1]{\textcolor[rgb]{0.00,0.23,0.31}{\textbf{#1}}}
\newcommand{\NormalTok}[1]{\textcolor[rgb]{0.00,0.23,0.31}{#1}}
\newcommand{\OperatorTok}[1]{\textcolor[rgb]{0.37,0.37,0.37}{#1}}
\newcommand{\OtherTok}[1]{\textcolor[rgb]{0.00,0.23,0.31}{#1}}
\newcommand{\PreprocessorTok}[1]{\textcolor[rgb]{0.68,0.00,0.00}{#1}}
\newcommand{\RegionMarkerTok}[1]{\textcolor[rgb]{0.00,0.23,0.31}{#1}}
\newcommand{\SpecialCharTok}[1]{\textcolor[rgb]{0.37,0.37,0.37}{#1}}
\newcommand{\SpecialStringTok}[1]{\textcolor[rgb]{0.13,0.47,0.30}{#1}}
\newcommand{\StringTok}[1]{\textcolor[rgb]{0.13,0.47,0.30}{#1}}
\newcommand{\VariableTok}[1]{\textcolor[rgb]{0.07,0.07,0.07}{#1}}
\newcommand{\VerbatimStringTok}[1]{\textcolor[rgb]{0.13,0.47,0.30}{#1}}
\newcommand{\WarningTok}[1]{\textcolor[rgb]{0.37,0.37,0.37}{\textit{#1}}}

\providecommand{\tightlist}{%
  \setlength{\itemsep}{0pt}\setlength{\parskip}{0pt}}\usepackage{longtable,booktabs,array}
\usepackage{calc} % for calculating minipage widths
% Correct order of tables after \paragraph or \subparagraph
\usepackage{etoolbox}
\makeatletter
\patchcmd\longtable{\par}{\if@noskipsec\mbox{}\fi\par}{}{}
\makeatother
% Allow footnotes in longtable head/foot
\IfFileExists{footnotehyper.sty}{\usepackage{footnotehyper}}{\usepackage{footnote}}
\makesavenoteenv{longtable}
\usepackage{graphicx}
\makeatletter
\def\maxwidth{\ifdim\Gin@nat@width>\linewidth\linewidth\else\Gin@nat@width\fi}
\def\maxheight{\ifdim\Gin@nat@height>\textheight\textheight\else\Gin@nat@height\fi}
\makeatother
% Scale images if necessary, so that they will not overflow the page
% margins by default, and it is still possible to overwrite the defaults
% using explicit options in \includegraphics[width, height, ...]{}
\setkeys{Gin}{width=\maxwidth,height=\maxheight,keepaspectratio}
% Set default figure placement to htbp
\makeatletter
\def\fps@figure{htbp}
\makeatother

\makeatletter
\@ifpackageloaded{caption}{}{\usepackage{caption}}
\AtBeginDocument{%
\ifdefined\contentsname
  \renewcommand*\contentsname{Table of contents}
\else
  \newcommand\contentsname{Table of contents}
\fi
\ifdefined\listfigurename
  \renewcommand*\listfigurename{List of Figures}
\else
  \newcommand\listfigurename{List of Figures}
\fi
\ifdefined\listtablename
  \renewcommand*\listtablename{List of Tables}
\else
  \newcommand\listtablename{List of Tables}
\fi
\ifdefined\figurename
  \renewcommand*\figurename{Figure}
\else
  \newcommand\figurename{Figure}
\fi
\ifdefined\tablename
  \renewcommand*\tablename{Table}
\else
  \newcommand\tablename{Table}
\fi
}
\@ifpackageloaded{float}{}{\usepackage{float}}
\floatstyle{ruled}
\@ifundefined{c@chapter}{\newfloat{codelisting}{h}{lop}}{\newfloat{codelisting}{h}{lop}[chapter]}
\floatname{codelisting}{Listing}
\newcommand*\listoflistings{\listof{codelisting}{List of Listings}}
\makeatother
\makeatletter
\makeatother
\makeatletter
\@ifpackageloaded{caption}{}{\usepackage{caption}}
\@ifpackageloaded{subcaption}{}{\usepackage{subcaption}}
\makeatother


% Times new roman-like
\usepackage{newtxtext,newtxmath}
\usepackage{sourcesanspro} % sans font
\usepackage{bm}

\providecommand{\tightlist}{%
  \setlength{\itemsep}{0pt}\setlength{\parskip}{2pt}}
\setcounter{secnumdepth}{0}

\usepackage{fancyhdr}
\usepackage{lastpage}
\pagestyle{fancy}
\fancyhf{}
  \renewcommand{\headrulewidth}{0pt}%
\fancyhead{} % clear all header fields
\fancyfoot{} % clear all footer fields
\fancyfoot[LE,LO]{\MakeUppercase{PSIM-80}}
\fancyfoot[CO,CE]{Page \textbf{\thepage} of \textbf{\pageref{LastPage}}}

\usepackage[hidelinks]{hyperref}

\usepackage{enumitem}
\setlist[description]{font=\bfseries\rmfamily,leftmargin=3.8cm,
    style=multiline,itemsep=1\baselineskip,parsep=2pt}
\setlist[enumerate]{font=\bfseries\rmfamily,leftmargin=1.2em,itemsep=1\baselineskip,parsep=2pt}

\usepackage{xstring}
% \usepackage{etoolbox}
% \usepackage{ifthen}

% special case 1 mark to be singular
\newcommand*{\rmarkcases}[1]{\IfStrEq{#1}{1}{[#1 mark]}{[#1 marks]}}%

\providecommand{\rmark}[1]{%
\begin{flushright}%
  \textbf{\rmarkcases{#1}}%
\end{flushright}%
}
% inline version if needed
\providecommand{\rmarkinline}[1]{\mbox{~}\hfill\mbox{\textbf{\rmarkcases{#1}}}}

\providecommand{\thisistheend}{\vfill{\hbox to \textwidth{\hfil * * * * * * * * * * * * * * *\hfil}}\vfill}

\input{_macros.tex}

\begin{document}
\pagecolor{white}
%
\begin{titlepage}
\thispagestyle{fancy}
\setcounter{page}{1}
\begin{center}
%
\vskip-2cm
\hbox to \hsize{
\makebox[50mm]{\hskip-6ex{\footnotesize \bfseries Apellidos:}\dotfill}\hspace{3ex}
\makebox[55mm]{{\footnotesize \bfseries Nombre:}\dotfill}\hspace{5ex}
\makebox[55mm]{{\footnotesize \bfseries Código:}\dotfill}
}
\vskip2em
%
\parbox[b]{80mm}{
\includegraphics[width=80mm]{../../recursos/imagenes/generales/Escuela_Rosario_logo.png}
}\par\vskip2em
\bfseries\large{UNIVERSIDAD ESCUELA COLOMBIANA DE INGENIERÍA}\\
\bfseries\large{SEMESTRE: \MakeUppercase{2024} -- \MakeUppercase{2}}\\[2em]
\setlength{\fboxrule}{1pt}\setlength{\fboxsep}{1em}
\framebox{%
\begin{minipage}{80mm}\centering
\MakeUppercase{PSIM-80}\\[0.5em]	% e.g. {COMP 102}
\MakeUppercase{Procesamiento de Señales e Imágenes Médicas}\\[0.5em]	
\MakeUppercase{Oct 21, 2024}
\end{minipage}}
%
\vspace{3\baselineskip}
\end{center}
%
\begin{description}
  \item[Tiempo Permitido:] {\bfseries \MakeUppercase{Una Hora.}}
  \item[Material Permitido:] {\bfseries \MakeUppercase{Apuntes con
caligrafía propia.}}\\
  NO se permite comunciación con compañeros ni préstamo de elementos.
  \item[Instrucciones:] 
  Responda cada pregunta según las instrucciones de la sección

  El examen consta de un total de \textbf{50} puntos.
\end{description}
%
\end{titlepage}

\newpage % First question must not start on front page

\setcounter{page}{2} % titlepage messed with numbers
\subsection{Primera Sección: Preguntas con única respuesta (25
puntos)}\label{primera-secciuxf3n-preguntas-con-uxfanica-respuesta-25-puntos}

Responda las preguntas, teniendo en cuenta el siguiente fragmento de
código

\phantomsection\label{primera-secciuxf3n}
\begin{Shaded}
\begin{Highlighting}[]
\NormalTok{data\_imagen }\OperatorTok{=}\NormalTok{ pyimag1.dcmread(ruta)}
\NormalTok{image }\OperatorTok{=}\NormalTok{ data\_imagen.pixel\_array}
\NormalTok{pyimag3.imshow(image, cmap}\OperatorTok{=}\StringTok{"gray"}\NormalTok{)}
\NormalTok{pyimag3.axis(}\StringTok{"off"}\NormalTok{)}
\end{Highlighting}
\end{Shaded}

\begin{enumerate}
\tightlist
\item
  (5 Puntos). El resultado del código es:

  \begin{enumerate}
  \tightlist
  \item
    Un error de tipo, que informa que image no es de tipo float
  \item
    Un error de tipo, que informa que image no se puede convertir en
    float
  \item
    Despliega una imagen en escala de grises, la imagen se encuentra en
    \emph{ruta}.
  \item
    Ninguna de las anteriores.
  \end{enumerate}
\item
  (5 puntos) ¿Que tipo de imagen permite cargar este código?

  \begin{enumerate}
  \tightlist
  \item
    Una imagen dicom.
  \item
    Una imagen png.
  \item
    Una imagen nii.
  \item
    Ninguna de las anteriores.
  \end{enumerate}
\item
  (5 Puntos) Cuales son las características de la imagen en formato
  dicom?

  \begin{enumerate}
  \tightlist
  \item
    Es una imagen híbrida, es decir que además de la información visual
    también tiene metadatos del paciente
  \item
    Las imágenes dicom son muy utilizadas porque aseguran
    interoperabilidad entre sistemas.
  \item
    a y b son correctas.
  \item
    Ninguna de las anteriores
  \end{enumerate}
\item
  (5 Puntos) La extracción del identificador de paciente se hace
  utilizando el siguiente comando:

  \begin{enumerate}
  \tightlist
  \item
    \emph{nombre\_paciente = image.PatientID}
  \item
    \emph{nombre\_paciente = data\_imagen.PatientID}
  \item
    \emph{nombre\_paciente = pyimag1.PatientID}
  \item
    \emph{nombre\_paciente = pyimag3.PatientID}
  \end{enumerate}
\item
  (5 Puntos) Suponiendo un funcionamiento perfecto del código, ¿que es
  pyimag1?

  \begin{enumerate}
  \tightlist
  \item
    pyimag1 es la ruta donde se encuentra ubicada la imagen que se
    quiere desplegar.
  \item
    pyimag1 representa la librería de despliegue visual de python
    (matplotlib.pyplot).
  \item
    pyimag1 representa la librería de procesamiento de imágenes de
    python (opencv).
  \item
    Ninguna de las anteriores.
  \end{enumerate}
\end{enumerate}

\subsection{Segunda Sección: Preguntas con múlitple respuesta (25
puntos)}\label{segunda-secciuxf3n-preguntas-con-muxfalitple-respuesta-25-puntos}

Responda las siguientes preguntas a partir del código presentado abajo.
Recuerde que este código se complementa con el presentado previamente.

\phantomsection\label{segunda-secciuxf3n}
\begin{Shaded}
\begin{Highlighting}[]
\NormalTok{kernel1 }\OperatorTok{=}\NormalTok{ (}\DecValTok{1}\OperatorTok{/}\DecValTok{9}\NormalTok{)}\OperatorTok{*}\NormalTok{pyimag4.array([[}\DecValTok{1}\NormalTok{, }\DecValTok{1}\NormalTok{, }\DecValTok{1}\NormalTok{], [}\DecValTok{1}\NormalTok{, }\DecValTok{1}\NormalTok{, }\DecValTok{1}\NormalTok{], [}\DecValTok{1}\NormalTok{, }\DecValTok{1}\NormalTok{, }\DecValTok{1}\NormalTok{]])}
\NormalTok{conv1 }\OperatorTok{=}\NormalTok{ pyimag2.filter2D(image, ddepth}\OperatorTok{={-}}\DecValTok{1}\NormalTok{, kernel}\OperatorTok{=}\NormalTok{kernel1)}
\NormalTok{conv1\_normalized }\OperatorTok{=}\NormalTok{ pyimag2.normalize(conv1, }\VariableTok{None}\NormalTok{, }\DecValTok{0}\NormalTok{, }\DecValTok{255}\NormalTok{, pyimag2.NORM\_MINMAX)}
\NormalTok{pyimag3.imshow(conv1\_normalized, cmap}\OperatorTok{=}\StringTok{"gray"}\NormalTok{)}
\NormalTok{conv1\_normalized.}\BuiltInTok{max}\NormalTok{()}
\end{Highlighting}
\end{Shaded}

\begin{enumerate}
\tightlist
\item
  (10 Puntos) ¿Cual es la función del código?

  \begin{enumerate}
  \tightlist
  \item
    Aplicar un kernel de convolución a la imagen que se encuentra
    almacenada en la variable image.
  \item
    Crear un kernel de convolución para encontrar bordes en una imagen.
  \item
    Hacer una normalización para pasar de un profundidad de 16bits a un
    profundidad de 8bits.
  \item
    Ninguna de las anteriores
  \end{enumerate}
\item
  (5 Puntos) ¿Que representa pyimag2?

  \begin{enumerate}
  \tightlist
  \item
    Representa a la librería matplotlib.
  \item
    Representa a la librería opencv.
  \item
    Representa a la librería numpy.
  \item
    Representa a la librería pydicom.
  \end{enumerate}
\item
  (10 Puntos) ¿Cuál es la profundidad de pixel de las variables conv1 y
  conv1\_normalized, respectivamente?

  \begin{enumerate}
  \tightlist
  \item
    16 bit y 16 bit
  \item
    8 bit y 8 bit
  \item
    16 bit y 8 bit
  \item
    8 bit y 16 bit
  \end{enumerate}
\end{enumerate}

\end{document}
