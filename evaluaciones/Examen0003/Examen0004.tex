\documentclass[12pt,a4paper]{article}

\usepackage[textwidth=165mm,textheight=240mm, includehead, includefoot, top =1cm, bottom=1cm,nomarginpar]{geometry}
\setlength{\parindent}{0pt}
\setlength{\parskip}{6pt plus 2pt minus 1pt}
\setlength{\emergencystretch}{3em}  % prevent overfull lines

% xetex settings
% \usepackage{fontspec,xltxtra,xunicode}
% \defaultfontfeatures{Mapping=tex-text,Scale=MatchLowercase}
% \setsansfont[Mapping=tex-text,Ligatures={Common}, Numbers={Lining}]{Helvetica Neue}
% \setmainfont[Mapping=tex-text,Ligatures={Common}, Numbers={Lining}]{Times New Roman}
% \usepackage[math-style=TeX]{unicode-math}
% \setmathfont{STIX Two Math}

%\usepackage{graphicx}
% bring in stuff from pandoc


\providecommand{\tightlist}{%
  \setlength{\itemsep}{0pt}\setlength{\parskip}{0pt}}\usepackage{longtable,booktabs,array}
\usepackage{calc} % for calculating minipage widths
% Correct order of tables after \paragraph or \subparagraph
\usepackage{etoolbox}
\makeatletter
\patchcmd\longtable{\par}{\if@noskipsec\mbox{}\fi\par}{}{}
\makeatother
% Allow footnotes in longtable head/foot
\IfFileExists{footnotehyper.sty}{\usepackage{footnotehyper}}{\usepackage{footnote}}
\makesavenoteenv{longtable}
\usepackage{graphicx}
\makeatletter
\def\maxwidth{\ifdim\Gin@nat@width>\linewidth\linewidth\else\Gin@nat@width\fi}
\def\maxheight{\ifdim\Gin@nat@height>\textheight\textheight\else\Gin@nat@height\fi}
\makeatother
% Scale images if necessary, so that they will not overflow the page
% margins by default, and it is still possible to overwrite the defaults
% using explicit options in \includegraphics[width, height, ...]{}
\setkeys{Gin}{width=\maxwidth,height=\maxheight,keepaspectratio}
% Set default figure placement to htbp
\makeatletter
\def\fps@figure{htbp}
\makeatother

\makeatletter
\@ifpackageloaded{caption}{}{\usepackage{caption}}
\AtBeginDocument{%
\ifdefined\contentsname
  \renewcommand*\contentsname{Table of contents}
\else
  \newcommand\contentsname{Table of contents}
\fi
\ifdefined\listfigurename
  \renewcommand*\listfigurename{List of Figures}
\else
  \newcommand\listfigurename{List of Figures}
\fi
\ifdefined\listtablename
  \renewcommand*\listtablename{List of Tables}
\else
  \newcommand\listtablename{List of Tables}
\fi
\ifdefined\figurename
  \renewcommand*\figurename{Figure}
\else
  \newcommand\figurename{Figure}
\fi
\ifdefined\tablename
  \renewcommand*\tablename{Table}
\else
  \newcommand\tablename{Table}
\fi
}
\@ifpackageloaded{float}{}{\usepackage{float}}
\floatstyle{ruled}
\@ifundefined{c@chapter}{\newfloat{codelisting}{h}{lop}}{\newfloat{codelisting}{h}{lop}[chapter]}
\floatname{codelisting}{Listing}
\newcommand*\listoflistings{\listof{codelisting}{List of Listings}}
\makeatother
\makeatletter
\makeatother
\makeatletter
\@ifpackageloaded{caption}{}{\usepackage{caption}}
\@ifpackageloaded{subcaption}{}{\usepackage{subcaption}}
\makeatother


% Times new roman-like
\usepackage{newtxtext,newtxmath}
\usepackage{sourcesanspro} % sans font
\usepackage{bm}

\providecommand{\tightlist}{%
  \setlength{\itemsep}{0pt}\setlength{\parskip}{2pt}}
\setcounter{secnumdepth}{0}

\usepackage{fancyhdr}
\usepackage{lastpage}
\pagestyle{fancy}
\fancyhf{}
  \renewcommand{\headrulewidth}{0pt}%
\fancyhead{} % clear all header fields
\fancyfoot{} % clear all footer fields
\fancyfoot[LE,LO]{\MakeUppercase{PSIM-80}}
\fancyfoot[CO,CE]{Page \textbf{\thepage} of \textbf{\pageref{LastPage}}}

\usepackage[hidelinks]{hyperref}

\usepackage{enumitem}
\setlist[description]{font=\bfseries\rmfamily,leftmargin=3.8cm,
    style=multiline,itemsep=1\baselineskip,parsep=2pt}
\setlist[enumerate]{font=\bfseries\rmfamily,leftmargin=1.2em,itemsep=1\baselineskip,parsep=2pt}

\usepackage{xstring}
% \usepackage{etoolbox}
% \usepackage{ifthen}

% special case 1 mark to be singular
\newcommand*{\rmarkcases}[1]{\IfStrEq{#1}{1}{[#1 mark]}{[#1 marks]}}%

\providecommand{\rmark}[1]{%
\begin{flushright}%
  \textbf{\rmarkcases{#1}}%
\end{flushright}%
}
% inline version if needed
\providecommand{\rmarkinline}[1]{\mbox{~}\hfill\mbox{\textbf{\rmarkcases{#1}}}}

\providecommand{\thisistheend}{\vfill{\hbox to \textwidth{\hfil * * * * * * * * * * * * * * *\hfil}}\vfill}

\input{_macros.tex}

\begin{document}
\pagecolor{white}
%
\begin{titlepage}
\thispagestyle{fancy}
\setcounter{page}{1}
\begin{center}
%
\vskip-2cm
\hbox to \hsize{
\makebox[50mm]{\hskip-6ex{\footnotesize \bfseries Apellidos:}\dotfill}\hspace{3ex}
\makebox[55mm]{{\footnotesize \bfseries Nombre:}\dotfill}\hspace{5ex}
\makebox[55mm]{{\footnotesize \bfseries Código:}\dotfill}
}
\vskip2em
%
\parbox[b]{80mm}{
\includegraphics[width=80mm]{../../recursos/imagenes/generales/Escuela_Rosario_logo.png}
}\par\vskip2em
\bfseries\large{UNIVERSIDAD ESCUELA COLOMBIANA DE INGENIERÍA}\\
\bfseries\large{SEMESTRE: \MakeUppercase{2024} -- \MakeUppercase{2}}\\[2em]
\setlength{\fboxrule}{1pt}\setlength{\fboxsep}{1em}
\framebox{%
\begin{minipage}{80mm}\centering
\MakeUppercase{PSIM-80}\\[0.5em]	% e.g. {COMP 102}
\MakeUppercase{Procesamiento de Señales e Imágenes Médicas}\\[0.5em]	
\MakeUppercase{Oct 21, 2024}
\end{minipage}}
%
\vspace{3\baselineskip}
\end{center}
%
\begin{description}
  \item[Tiempo Permitido:] {\bfseries \MakeUppercase{Una Hora.}}
  \item[Material Permitido:] {\bfseries \MakeUppercase{Apuntes con
caligrafía propia.}}\\
  NO se permite comunciación con compañeros ni préstamo de elementos.
  \item[Instrucciones:] 
  Responda cada pregunta según las instrucciones de la sección

  El examen consta de un total de \textbf{50} puntos.
\end{description}
%
\end{titlepage}

\newpage % First question must not start on front page

\setcounter{page}{2} % titlepage messed with numbers
\begin{enumerate}
\item
  Erosión en morfología Contexto: Un patólogo necesita reducir el tamaño
  de pequeñas imperfecciones en una imagen digital de tejido tomada con
  microscopía. Pregunta: ¿Qué operación morfológica sería más adecuada
  para reducir las imperfecciones?

  \begin{enumerate}
  \tightlist
  \item
    Erosión.
  \item
    Dilatación.
  \item
    Transformada de Hough.
  \item
    Histograma igualado.
  \end{enumerate}
\item
  Histograma y contraste Contexto: Un técnico de imágenes médicas
  observa que una imagen de ultrasonido muestra un bajo contraste entre
  los tejidos debido a una mala distribución de los niveles de
  intensidad. Pregunta: ¿Qué herramienta de procesamiento de imágenes es
  la más adecuada para mejorar el contraste en esta situación? \#.
  Ecualización de histograma. \#. Filtro de mediana. \#. Filtro
  Laplaciano. \#. Umbralización global.
\item
  Aplicación de la convolución Contexto: En una imagen de resonancia
  magnética de una rodilla, se requiere un filtro que resalte las líneas
  y bordes de los huesos para evaluar posibles fracturas. Pregunta: ¿Qué
  operación de procesamiento de imágenes debe aplicarse a esta imagen?

  \begin{enumerate}
  \tightlist
  \item
    Convolución con un filtro de Sobel.
  \item
    Filtro de promedio.
  \item
    Ecualización de histograma.
  \item
    Transformada de Fourier.
  \end{enumerate}
\item
  Un técnico en imágenes está procesando una imagen de rayos X que
  contiene ruido ``sal y pimienta''. ¿Qué filtro sería el más adecuado
  para reducir este tipo de ruido en la imagen?

  \begin{enumerate}
  \tightlist
  \item
    Filtro de mediana.
  \item
    Filtro de promedio.
  \item
    Convolución con filtro Laplaciano.
  \item
    Transformada de Fourier.
  \end{enumerate}
\item
  Un radiólogo necesita comparar imágenes de tomografía computarizada
  tomadas en diferentes momentos para evaluar la progresión de un tumor
  cerebral. Sin embargo, las imágenes han sido capturadas con ángulos
  ligeramente diferentes. Para realizar una comparación precisa, el
  radiólogo decide rotar una de las imágenes, de manera que coincida con
  la otra sin distorsionar las proporciones originales del cerebro. ¿Qué
  tipo de transformación geométrica debe aplicar el radiólogo para rotar
  la imagen y mantener la coherencia de las proporciones?

  \begin{enumerate}
  \tightlist
  \item
    Transformación afín.
  \item
    Transformación de escala.
  \item
    Transformación bilineal.
  \item
    Transformación polar.
  \end{enumerate}
\item
  Umbralización adaptativa Contexto: Un investigador está analizando
  imágenes de microscopía con variaciones de iluminación y necesita
  segmentar las células de manera precisa. Pregunta: ¿Qué técnica de
  umbralización es más adecuada para manejar las variaciones de
  iluminación en la imagen?

  \begin{enumerate}
  \tightlist
  \item
    Filtro de Sobel.
  \item
    Convolución.
  \item
    Umbralización adaptativa.
  \item
    Transformada de Fourier.
  \end{enumerate}
\item
  Un radiólogo está trabajando con imágenes de mamografía para detectar
  posibles tumores. Desea separar las áreas de tejido sospechoso del
  fondo oscuro. ¿Qué técnica de umbralización debería usar para aislar
  las regiones más brillantes?

  \begin{enumerate}
  \tightlist
  \item
    Umbralización global.
  \item
    Filtro de mediana.
  \item
    Umbralización adaptativa.
  \item
    Filtrado de alta frecuencia.
  \end{enumerate}
\item
  Un radiólogo está utilizando mamografías para identificar áreas
  sospechosas de tejido. Para segmentar automáticamente las regiones de
  interés, decide aplicar una técnica de umbralización que maximice la
  separación entre el fondo y las áreas sospechosas. ¿Qué técnica de
  umbralización es la más adecuada para dividir la imagen en dos clases
  optimizadas de píxeles (fondo y tejido sospechoso)?

  \begin{enumerate}
  \tightlist
  \item
    Umbralización global.
  \item
    Umbralización de Otsu.
  \item
    Umbralización adaptativa.
  \item
    Filtro de mediana.
  \end{enumerate}
\end{enumerate}

\end{document}
