\documentclass[12pt,a4paper]{article}

\usepackage[textwidth=165mm,textheight=240mm, includehead, includefoot, top =1cm, bottom=1cm,nomarginpar]{geometry}
\setlength{\parindent}{0pt}
\setlength{\parskip}{6pt plus 2pt minus 1pt}
\setlength{\emergencystretch}{3em}  % prevent overfull lines

% xetex settings
% \usepackage{fontspec,xltxtra,xunicode}
% \defaultfontfeatures{Mapping=tex-text,Scale=MatchLowercase}
% \setsansfont[Mapping=tex-text,Ligatures={Common}, Numbers={Lining}]{Helvetica Neue}
% \setmainfont[Mapping=tex-text,Ligatures={Common}, Numbers={Lining}]{Times New Roman}
% \usepackage[math-style=TeX]{unicode-math}
% \setmathfont{STIX Two Math}

%\usepackage{graphicx}
% bring in stuff from pandoc


\providecommand{\tightlist}{%
  \setlength{\itemsep}{0pt}\setlength{\parskip}{0pt}}\usepackage{longtable,booktabs,array}
\usepackage{calc} % for calculating minipage widths
% Correct order of tables after \paragraph or \subparagraph
\usepackage{etoolbox}
\makeatletter
\patchcmd\longtable{\par}{\if@noskipsec\mbox{}\fi\par}{}{}
\makeatother
% Allow footnotes in longtable head/foot
\IfFileExists{footnotehyper.sty}{\usepackage{footnotehyper}}{\usepackage{footnote}}
\makesavenoteenv{longtable}
\usepackage{graphicx}
\makeatletter
\def\maxwidth{\ifdim\Gin@nat@width>\linewidth\linewidth\else\Gin@nat@width\fi}
\def\maxheight{\ifdim\Gin@nat@height>\textheight\textheight\else\Gin@nat@height\fi}
\makeatother
% Scale images if necessary, so that they will not overflow the page
% margins by default, and it is still possible to overwrite the defaults
% using explicit options in \includegraphics[width, height, ...]{}
\setkeys{Gin}{width=\maxwidth,height=\maxheight,keepaspectratio}
% Set default figure placement to htbp
\makeatletter
\def\fps@figure{htbp}
\makeatother

\makeatletter
\@ifpackageloaded{caption}{}{\usepackage{caption}}
\AtBeginDocument{%
\ifdefined\contentsname
  \renewcommand*\contentsname{Table of contents}
\else
  \newcommand\contentsname{Table of contents}
\fi
\ifdefined\listfigurename
  \renewcommand*\listfigurename{List of Figures}
\else
  \newcommand\listfigurename{List of Figures}
\fi
\ifdefined\listtablename
  \renewcommand*\listtablename{List of Tables}
\else
  \newcommand\listtablename{List of Tables}
\fi
\ifdefined\figurename
  \renewcommand*\figurename{Figure}
\else
  \newcommand\figurename{Figure}
\fi
\ifdefined\tablename
  \renewcommand*\tablename{Table}
\else
  \newcommand\tablename{Table}
\fi
}
\@ifpackageloaded{float}{}{\usepackage{float}}
\floatstyle{ruled}
\@ifundefined{c@chapter}{\newfloat{codelisting}{h}{lop}}{\newfloat{codelisting}{h}{lop}[chapter]}
\floatname{codelisting}{Listing}
\newcommand*\listoflistings{\listof{codelisting}{List of Listings}}
\makeatother
\makeatletter
\makeatother
\makeatletter
\@ifpackageloaded{caption}{}{\usepackage{caption}}
\@ifpackageloaded{subcaption}{}{\usepackage{subcaption}}
\makeatother


% Times new roman-like
\usepackage{newtxtext,newtxmath}
\usepackage{sourcesanspro} % sans font
\usepackage{bm}

\providecommand{\tightlist}{%
  \setlength{\itemsep}{0pt}\setlength{\parskip}{2pt}}
\setcounter{secnumdepth}{0}

\usepackage{fancyhdr}
\usepackage{lastpage}
\pagestyle{fancy}
\fancyhf{}
  \renewcommand{\headrulewidth}{0pt}%
\fancyhead{} % clear all header fields
\fancyfoot{} % clear all footer fields
\fancyfoot[LE,LO]{\MakeUppercase{PSIM-80}}
\fancyfoot[CO,CE]{Page \textbf{\thepage} of \textbf{\pageref{LastPage}}}

\usepackage[hidelinks]{hyperref}

\usepackage{enumitem}
\setlist[description]{font=\bfseries\rmfamily,leftmargin=3.8cm,
    style=multiline,itemsep=1\baselineskip,parsep=2pt}
\setlist[enumerate]{font=\bfseries\rmfamily,leftmargin=1.2em,itemsep=1\baselineskip,parsep=2pt}

\usepackage{xstring}
% \usepackage{etoolbox}
% \usepackage{ifthen}

% special case 1 mark to be singular
\newcommand*{\rmarkcases}[1]{\IfStrEq{#1}{1}{[#1 mark]}{[#1 marks]}}%

\providecommand{\rmark}[1]{%
\begin{flushright}%
  \textbf{\rmarkcases{#1}}%
\end{flushright}%
}
% inline version if needed
\providecommand{\rmarkinline}[1]{\mbox{~}\hfill\mbox{\textbf{\rmarkcases{#1}}}}

\providecommand{\thisistheend}{\vfill{\hbox to \textwidth{\hfil * * * * * * * * * * * * * * *\hfil}}\vfill}

\input{_macros.tex}

\begin{document}
\pagecolor{white}
%
\begin{titlepage}
\thispagestyle{fancy}
\setcounter{page}{1}
\begin{center}
%
\vskip-2cm
\hbox to \hsize{
\makebox[50mm]{\hskip-6ex{\footnotesize \bfseries Apellidos:}\dotfill}\hspace{3ex}
\makebox[55mm]{{\footnotesize \bfseries Nombre:}\dotfill}\hspace{5ex}
\makebox[55mm]{{\footnotesize \bfseries Código:}\dotfill}
}
\vskip2em
%
\parbox[b]{80mm}{
\includegraphics[width=80mm]{../../recursos/imagenes/generales/Escuela_Rosario_logo.png}
}\par\vskip2em
\bfseries\large{UNIVERSIDAD ESCUELA COLOMBIANA DE INGENIERÍA}\\
\bfseries\large{SEMESTRE: \MakeUppercase{2024} -- \MakeUppercase{2}}\\[2em]
\setlength{\fboxrule}{1pt}\setlength{\fboxsep}{1em}
\framebox{%
\begin{minipage}{80mm}\centering
\MakeUppercase{PSIM-80}\\[0.5em]	% e.g. {COMP 102}
\MakeUppercase{Procesamiento de Señales e Imágenes Médicas}\\[0.5em]	
\MakeUppercase{Oct 21, 2024}
\end{minipage}}
%
\vspace{3\baselineskip}
\end{center}
%
\begin{description}
  \item[Tiempo Permitido:] {\bfseries \MakeUppercase{Una Hora.}}
  \item[Material Permitido:] {\bfseries \MakeUppercase{Apuntes con
caligrafía propia.}}\\
  NO se permite comunciación con compañeros ni préstamo de elementos.
  \item[Instrucciones:] 
  Responda cada pregunta según las instrucciones de la sección

  El examen consta de un total de \textbf{50} puntos.
\end{description}
%
\end{titlepage}

\newpage % First question must not start on front page

\setcounter{page}{2} % titlepage messed with numbers
\begin{enumerate}
\tightlist
\item
  Resolución espacial en imágenes médicas Contexto: En una resonancia
  magnética, un radiólogo necesita distinguir pequeñas estructuras
  cerebrales en una imagen de baja resolución espacial. Pregunta: ¿Qué
  parámetro debería mejorar el radiólogo para aumentar la capacidad de
  distinguir detalles finos en la imagen?

  \begin{enumerate}
  \tightlist
  \item
    Tamaño del píxel.
  \item
    Resolución espacial.
  \item
    Profundidad de color.
  \item
    Relación señal-ruido.
  \end{enumerate}
\item
  Filtraje en el dominio del espacio Contexto: Un técnico de imágenes
  médicas está procesando una imagen de ultrasonido con ruido y decide
  utilizar un filtro espacial para reducir el ruido sin perder detalles
  importantes de los tejidos. Pregunta: ¿Qué tipo de filtro espacial es
  más adecuado para esta tarea?

  \begin{enumerate}
  \tightlist
  \item
    Filtro de mediana.
  \item
    Filtro Laplaciano.
  \item
    Histograma igualado.
  \item
    Filtro de Sobel.
  \end{enumerate}
\item
  Filtro de suavizado Contexto: En una imagen de rayos X, un
  especialista en procesamiento de imágenes desea reducir el ruido
  aleatorio sin afectar en gran medida los detalles finos, como
  fracturas. Pregunta: ¿Qué filtro es más adecuado para suavizar la
  imagen?

  \begin{enumerate}
  \tightlist
  \item
    Filtro de promedio.
  \item
    Filtro Laplaciano.
  \item
    Filtro de Sobel.
  \item
    Filtro de alta frecuencia.
  \end{enumerate}
\item
  Convolución en imágenes biomédicas Contexto: En una imagen de
  resonancia magnética, el análisis de la estructura cerebral requiere
  mejorar los bordes de las regiones de interés. Pregunta: ¿Qué
  operación se debe aplicar a la imagen para realizar este realce de
  bordes?

  \begin{enumerate}
  \tightlist
  \item
    Transformada de Fourier.
  \item
    Convolución con un filtro de bordes.
  \item
    Histograma de la imagen.
  \item
    Segmentación de regiones.
  \end{enumerate}
\item
  Histograma de intensidad Contexto: En una imagen de tomografía de
  rayos X, un médico nota que la mayoría de los píxeles se concentran en
  niveles bajos de intensidad, lo que genera una imagen oscura.
  Pregunta: ¿Qué técnica podría aplicar para redistribuir los niveles de
  intensidad y mejorar la visibilidad de los detalles?

  \begin{enumerate}
  \tightlist
  \item
    Filtro de mediana.
  \item
    Ecualización de histograma.
  \item
    Transformada de Fourier.
  \item
    Umbralización adaptativa.
  \end{enumerate}
\item
  Erosión en morfología Contexto: Un patólogo necesita reducir el tamaño
  de pequeñas imperfecciones en una imagen digital de tejido tomada con
  microscopía. Pregunta: ¿Qué operación morfológica sería más adecuada
  para reducir las imperfecciones?

  \begin{enumerate}
  \tightlist
  \item
    Erosión.
  \item
    Dilatación.
  \item
    Transformada de Hough.
  \item
    Histograma igualado.
  \end{enumerate}
\item
  Filtros de alta frecuencia Contexto: Un médico necesita mejorar la
  visualización de los bordes de los tumores en una imagen de tomografía
  computarizada para hacer un diagnóstico más preciso. Pregunta: ¿Qué
  tipo de filtro espacial sería más adecuado para resaltar los bordes?

  \begin{enumerate}
  \tightlist
  \item
    Filtro de promedio.
  \item
    Filtro de alta frecuencia.
  \item
    Filtro de mediana.
  \item
    Umbralización adaptativa.
  \end{enumerate}
\item
  Operaciones morfológicas para eliminar ruido Contexto: En una imagen
  de microscopía de células, se observan pequeños puntos aislados de
  ruido después de la segmentación. Pregunta: ¿Qué operación morfológica
  es más adecuada para eliminar estos puntos de ruido?

  \begin{enumerate}
  \tightlist
  \item
    Apertura.
  \item
    Dilatación.
  \item
    Filtro de Sobel.
  \item
    Transformada gamma.
  \end{enumerate}
\end{enumerate}

\end{document}
