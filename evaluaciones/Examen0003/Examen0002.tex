\documentclass[12pt,a4paper]{article}

\usepackage[textwidth=165mm,textheight=240mm, includehead, includefoot, top =1cm, bottom=1cm,nomarginpar]{geometry}
\setlength{\parindent}{0pt}
\setlength{\parskip}{6pt plus 2pt minus 1pt}
\setlength{\emergencystretch}{3em}  % prevent overfull lines

% xetex settings
% \usepackage{fontspec,xltxtra,xunicode}
% \defaultfontfeatures{Mapping=tex-text,Scale=MatchLowercase}
% \setsansfont[Mapping=tex-text,Ligatures={Common}, Numbers={Lining}]{Helvetica Neue}
% \setmainfont[Mapping=tex-text,Ligatures={Common}, Numbers={Lining}]{Times New Roman}
% \usepackage[math-style=TeX]{unicode-math}
% \setmathfont{STIX Two Math}

%\usepackage{graphicx}
% bring in stuff from pandoc


\providecommand{\tightlist}{%
  \setlength{\itemsep}{0pt}\setlength{\parskip}{0pt}}\usepackage{longtable,booktabs,array}
\usepackage{calc} % for calculating minipage widths
% Correct order of tables after \paragraph or \subparagraph
\usepackage{etoolbox}
\makeatletter
\patchcmd\longtable{\par}{\if@noskipsec\mbox{}\fi\par}{}{}
\makeatother
% Allow footnotes in longtable head/foot
\IfFileExists{footnotehyper.sty}{\usepackage{footnotehyper}}{\usepackage{footnote}}
\makesavenoteenv{longtable}
\usepackage{graphicx}
\makeatletter
\def\maxwidth{\ifdim\Gin@nat@width>\linewidth\linewidth\else\Gin@nat@width\fi}
\def\maxheight{\ifdim\Gin@nat@height>\textheight\textheight\else\Gin@nat@height\fi}
\makeatother
% Scale images if necessary, so that they will not overflow the page
% margins by default, and it is still possible to overwrite the defaults
% using explicit options in \includegraphics[width, height, ...]{}
\setkeys{Gin}{width=\maxwidth,height=\maxheight,keepaspectratio}
% Set default figure placement to htbp
\makeatletter
\def\fps@figure{htbp}
\makeatother

\makeatletter
\@ifpackageloaded{caption}{}{\usepackage{caption}}
\AtBeginDocument{%
\ifdefined\contentsname
  \renewcommand*\contentsname{Table of contents}
\else
  \newcommand\contentsname{Table of contents}
\fi
\ifdefined\listfigurename
  \renewcommand*\listfigurename{List of Figures}
\else
  \newcommand\listfigurename{List of Figures}
\fi
\ifdefined\listtablename
  \renewcommand*\listtablename{List of Tables}
\else
  \newcommand\listtablename{List of Tables}
\fi
\ifdefined\figurename
  \renewcommand*\figurename{Figure}
\else
  \newcommand\figurename{Figure}
\fi
\ifdefined\tablename
  \renewcommand*\tablename{Table}
\else
  \newcommand\tablename{Table}
\fi
}
\@ifpackageloaded{float}{}{\usepackage{float}}
\floatstyle{ruled}
\@ifundefined{c@chapter}{\newfloat{codelisting}{h}{lop}}{\newfloat{codelisting}{h}{lop}[chapter]}
\floatname{codelisting}{Listing}
\newcommand*\listoflistings{\listof{codelisting}{List of Listings}}
\makeatother
\makeatletter
\makeatother
\makeatletter
\@ifpackageloaded{caption}{}{\usepackage{caption}}
\@ifpackageloaded{subcaption}{}{\usepackage{subcaption}}
\makeatother


% Times new roman-like
\usepackage{newtxtext,newtxmath}
\usepackage{sourcesanspro} % sans font
\usepackage{bm}

\providecommand{\tightlist}{%
  \setlength{\itemsep}{0pt}\setlength{\parskip}{2pt}}
\setcounter{secnumdepth}{0}

\usepackage{fancyhdr}
\usepackage{lastpage}
\pagestyle{fancy}
\fancyhf{}
  \renewcommand{\headrulewidth}{0pt}%
\fancyhead{} % clear all header fields
\fancyfoot{} % clear all footer fields
\fancyfoot[LE,LO]{\MakeUppercase{PSIM-80}}
\fancyfoot[CO,CE]{Page \textbf{\thepage} of \textbf{\pageref{LastPage}}}

\usepackage[hidelinks]{hyperref}

\usepackage{enumitem}
\setlist[description]{font=\bfseries\rmfamily,leftmargin=3.8cm,
    style=multiline,itemsep=1\baselineskip,parsep=2pt}
\setlist[enumerate]{font=\bfseries\rmfamily,leftmargin=1.2em,itemsep=1\baselineskip,parsep=2pt}

\usepackage{xstring}
% \usepackage{etoolbox}
% \usepackage{ifthen}

% special case 1 mark to be singular
\newcommand*{\rmarkcases}[1]{\IfStrEq{#1}{1}{[#1 mark]}{[#1 marks]}}%

\providecommand{\rmark}[1]{%
\begin{flushright}%
  \textbf{\rmarkcases{#1}}%
\end{flushright}%
}
% inline version if needed
\providecommand{\rmarkinline}[1]{\mbox{~}\hfill\mbox{\textbf{\rmarkcases{#1}}}}

\providecommand{\thisistheend}{\vfill{\hbox to \textwidth{\hfil * * * * * * * * * * * * * * *\hfil}}\vfill}

\input{_macros.tex}

\begin{document}
\pagecolor{white}
%
\begin{titlepage}
\thispagestyle{fancy}
\setcounter{page}{1}
\begin{center}
%
\vskip-2cm
\hbox to \hsize{
\makebox[50mm]{\hskip-6ex{\footnotesize \bfseries Apellidos:}\dotfill}\hspace{3ex}
\makebox[55mm]{{\footnotesize \bfseries Nombre:}\dotfill}\hspace{5ex}
\makebox[55mm]{{\footnotesize \bfseries Código:}\dotfill}
}
\vskip2em
%
\parbox[b]{80mm}{
\includegraphics[width=80mm]{../../recursos/imagenes/generales/Escuela_Rosario_logo.png}
}\par\vskip2em
\bfseries\large{UNIVERSIDAD ESCUELA COLOMBIANA DE INGENIERÍA}\\
\bfseries\large{SEMESTRE: \MakeUppercase{2024} -- \MakeUppercase{2}}\\[2em]
\setlength{\fboxrule}{1pt}\setlength{\fboxsep}{1em}
\framebox{%
\begin{minipage}{80mm}\centering
\MakeUppercase{PSIM-80}\\[0.5em]	% e.g. {COMP 102}
\MakeUppercase{Procesamiento de Señales e Imágenes Médicas}\\[0.5em]	
\MakeUppercase{Oct 21, 2024}
\end{minipage}}
%
\vspace{3\baselineskip}
\end{center}
%
\begin{description}
  \item[Tiempo Permitido:] {\bfseries \MakeUppercase{Una Hora.}}
  \item[Material Permitido:] {\bfseries \MakeUppercase{Apuntes con
caligrafía propia.}}\\
  NO se permite comunciación con compañeros ni préstamo de elementos.
  \item[Instrucciones:] 
  Responda cada pregunta según las instrucciones de la sección

  El examen consta de un total de \textbf{50} puntos.
\end{description}
%
\end{titlepage}

\newpage % First question must not start on front page

\setcounter{page}{2} % titlepage messed with numbers
\begin{enumerate}
\tightlist
\item
  Durante una exploración por tomografía computarizada (TC) de cráneo,
  un paciente con implantes metálicos genera artefactos que distorsionan
  la imagen alrededor de los implantes, dificultando el diagnóstico.
  Estos artefactos metálicos producen líneas brillantes o sombras que
  interfieren con la claridad de la imagen. ¿Qué técnica de
  procesamiento de imágenes puede ayudar a reducir los artefactos
  metálicos en la tomografía computarizada?

  \begin{enumerate}
  \tightlist
  \item
    Filtrado adaptativo.
  \item
    Registro de imágenes.
  \item
    Transformada de Fourier.
  \item
    Erosión morfológica.
  \end{enumerate}
\item
  En el análisis de mamografías, es fundamental detectar
  microcalcificaciones que pueden ser indicativas de cáncer de mama en
  etapa temprana. Las microcalcificaciones son pequeñas áreas de alta
  intensidad en la imagen, pero debido a su tamaño reducido y al ruido
  de fondo, son difíciles de detectar visualmente.¿Qué técnica es más
  adecuada para resaltar las microcalcificaciones en una mamografía?

  \begin{enumerate}
  \tightlist
  \item
    Filtro de Sobel.
  \item
    Filtro de mediana.
  \item
    Filtro de alta frecuencia.
  \item
    Transformada de promedio.
  \end{enumerate}
\item
  Un radiólogo está revisando imágenes de rayos X de una fractura ósea.
  Sin embargo, el bajo contraste entre el hueso y los tejidos
  circundantes dificulta la identificación precisa de la fractura. Para
  mejorar la visibilidad de los detalles en las imágenes, el radiólogo
  decide aplicar un método de realce de contraste.¿Qué técnica de
  procesamiento de imágenes sería la más adecuada para mejorar el
  contraste en esta imagen de rayos X?

  \begin{enumerate}
  \tightlist
  \item
    Histograma igualado.
  \item
    Filtro Laplaciano.
  \item
    Transformada de Fourier.
  \item
    Registro de imágenes.
  \end{enumerate}
\item
  Un radiólogo necesita comparar imágenes de tomografía computarizada
  tomadas en diferentes momentos para evaluar la progresión de un tumor
  cerebral. Sin embargo, las imágenes han sido capturadas con ángulos
  ligeramente diferentes. Para realizar una comparación precisa, el
  radiólogo decide rotar una de las imágenes, de manera que coincida con
  la otra sin distorsionar las proporciones originales del cerebro. ¿Qué
  tipo de transformación geométrica debe aplicar el radiólogo para rotar
  la imagen y mantener la coherencia de las proporciones?

  \begin{enumerate}
  \tightlist
  \item
    Transformación afín.
  \item
    Transformación de escala.
  \item
    Transformación bilineal.
  \item
    Transformación polar.
  \end{enumerate}
\item
  Un investigador en imágenes biomédicas está procesando imágenes
  microscópicas de muestras de tejido. Muchas de estas imágenes
  contienen pequeños objetos no deseados o ruido generado durante la
  adquisición. El investigador necesita aplicar una técnica morfológica
  que elimine estos objetos pequeños y ruidosos, pero manteniendo
  intactas las estructuras más grandes que son de interés para el
  análisis. ¿Qué operación morfológica es la más adecuada para este
  propósito en una imagen binaria?

  \begin{enumerate}
  \tightlist
  \item
    Erosión.
  \item
    Dilatación.
  \item
    Apertura.
  \item
    Cierre.
  \end{enumerate}
\item
  Un médico está utilizando un software de análisis de imágenes de
  ultrasonido para evaluar el contorno de arterias y venas en pacientes
  con sospecha de enfermedad cardiovascular. Para ello, es fundamental
  identificar claramente los bordes de las estructuras vasculares y
  diferenciar entre las arterias afectadas y el tejido circundante. El
  software debe aplicar un algoritmo de detección de bordes que permita
  delinear con precisión estas estructuras. ¿Cuál de las siguientes
  técnicas es más adecuada para detectar los bordes de las arterias y
  venas en las imágenes de ultrasonido?

  \begin{enumerate}
  \tightlist
  \item
    Filtro de mediana.
  \item
    Transformada de Fourier.
  \item
    Operador de Sobel.
  \item
    Filtro gaussiano.
  \end{enumerate}
\item
  Un especialista en imágenes médicas trabaja con imágenes obtenidas por
  microscopía para identificar células cancerosas en muestras de tejido.
  La diferencia de intensidad entre las células y el fondo de la imagen
  permite aplicar una técnica para binarizar la imagen, separando las
  células del fondo oscuro. Esta técnica ayuda a cuantificar las células
  presentes en la muestra de manera automática. ¿Cómo se define la
  umbralización en este contexto y cuál es su función principal?

  \begin{enumerate}
  \tightlist
  \item
    Un método para suavizar las transiciones entre diferentes niveles de
    color en la imagen.
  \item
    Un proceso que ajusta los niveles de color para mejorar la
    visibilidad de los detalles.
  \item
    Un procedimiento para convertir una imagen en binaria en función de
    un valor de intensidad, separando los objetos de interés del fondo.
  \item
    Un algoritmo para detectar bordes mediante gradientes de color.
  \end{enumerate}
\item
  Un equipo médico está utilizando cámaras especializadas para capturar
  imágenes laparoscópicas durante cirugías mínimamente invasivas. Sin
  embargo, en muchas ocasiones las imágenes se ven afectadas por ruido
  tipo ``sal y pimienta'' debido a interferencias electrónicas. El
  equipo de ingeniería biomédica necesita eliminar este ruido sin perder
  los detalles finos, que son cruciales para la correcta identificación
  de estructuras internas. Pregunta: ¿Cuál es el filtro más adecuado
  para eliminar el ruido ``sal y pimienta'' sin afectar
  significativamente la nitidez de los detalles quirúrgicos en la
  imagen?

  \begin{enumerate}
  \tightlist
  \item
    Filtro de mediana.
  \item
    Filtro de Canny.
  \item
    Filtro de Sobel.
  \item
    Filtro Laplaciano.
  \end{enumerate}
\end{enumerate}

\end{document}
