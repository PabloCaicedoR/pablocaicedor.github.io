% Options for packages loaded elsewhere
% Options for packages loaded elsewhere
\PassOptionsToPackage{unicode}{hyperref}
\PassOptionsToPackage{hyphens}{url}
%
\documentclass[
  ignorenonframetext,
  aspectratio=32,
  spanish,
]{beamer}
\newif\ifbibliography
\usepackage{pgfpages}
\setbeamertemplate{caption}[numbered]
\setbeamertemplate{caption label separator}{: }
\setbeamercolor{caption name}{fg=normal text.fg}
\beamertemplatenavigationsymbolshorizontal
% remove section numbering
\setbeamertemplate{part page}{
  \centering
  \begin{beamercolorbox}[sep=16pt,center]{part title}
    \usebeamerfont{part title}\insertpart\par
  \end{beamercolorbox}
}
\setbeamertemplate{section page}{
  \centering
  \begin{beamercolorbox}[sep=12pt,center]{section title}
    \usebeamerfont{section title}\insertsection\par
  \end{beamercolorbox}
}
\setbeamertemplate{subsection page}{
  \centering
  \begin{beamercolorbox}[sep=8pt,center]{subsection title}
    \usebeamerfont{subsection title}\insertsubsection\par
  \end{beamercolorbox}
}
% Prevent slide breaks in the middle of a paragraph
\widowpenalties 1 10000
\raggedbottom
\AtBeginPart{
  \frame{\partpage}
}
\AtBeginSection{
  \ifbibliography
  \else
    \frame{\sectionpage}
  \fi
}
\AtBeginSubsection{
  \frame{\subsectionpage}
}
\usepackage{iftex}
\ifPDFTeX
  \usepackage[T1]{fontenc}
  \usepackage[utf8]{inputenc}
  \usepackage{textcomp} % provide euro and other symbols
\else % if luatex or xetex
  \usepackage{unicode-math} % this also loads fontspec
  \defaultfontfeatures{Scale=MatchLowercase}
  \defaultfontfeatures[\rmfamily]{Ligatures=TeX,Scale=1}
\fi
\usepackage{lmodern}

\usetheme[]{CambridgeUS}
\ifPDFTeX\else
  % xetex/luatex font selection
\fi
% Use upquote if available, for straight quotes in verbatim environments
\IfFileExists{upquote.sty}{\usepackage{upquote}}{}
\IfFileExists{microtype.sty}{% use microtype if available
  \usepackage[]{microtype}
  \UseMicrotypeSet[protrusion]{basicmath} % disable protrusion for tt fonts
}{}
\makeatletter
\@ifundefined{KOMAClassName}{% if non-KOMA class
  \IfFileExists{parskip.sty}{%
    \usepackage{parskip}
  }{% else
    \setlength{\parindent}{0pt}
    \setlength{\parskip}{6pt plus 2pt minus 1pt}}
}{% if KOMA class
  \KOMAoptions{parskip=half}}
\makeatother

\usepackage{color}
\usepackage{fancyvrb}
\newcommand{\VerbBar}{|}
\newcommand{\VERB}{\Verb[commandchars=\\\{\}]}
\DefineVerbatimEnvironment{Highlighting}{Verbatim}{commandchars=\\\{\}}
% Add ',fontsize=\small' for more characters per line
\usepackage{framed}
\definecolor{shadecolor}{RGB}{241,243,245}
\newenvironment{Shaded}{\begin{snugshade}}{\end{snugshade}}
\newcommand{\AlertTok}[1]{\textcolor[rgb]{0.68,0.00,0.00}{#1}}
\newcommand{\AnnotationTok}[1]{\textcolor[rgb]{0.37,0.37,0.37}{#1}}
\newcommand{\AttributeTok}[1]{\textcolor[rgb]{0.40,0.45,0.13}{#1}}
\newcommand{\BaseNTok}[1]{\textcolor[rgb]{0.68,0.00,0.00}{#1}}
\newcommand{\BuiltInTok}[1]{\textcolor[rgb]{0.00,0.23,0.31}{#1}}
\newcommand{\CharTok}[1]{\textcolor[rgb]{0.13,0.47,0.30}{#1}}
\newcommand{\CommentTok}[1]{\textcolor[rgb]{0.37,0.37,0.37}{#1}}
\newcommand{\CommentVarTok}[1]{\textcolor[rgb]{0.37,0.37,0.37}{\textit{#1}}}
\newcommand{\ConstantTok}[1]{\textcolor[rgb]{0.56,0.35,0.01}{#1}}
\newcommand{\ControlFlowTok}[1]{\textcolor[rgb]{0.00,0.23,0.31}{\textbf{#1}}}
\newcommand{\DataTypeTok}[1]{\textcolor[rgb]{0.68,0.00,0.00}{#1}}
\newcommand{\DecValTok}[1]{\textcolor[rgb]{0.68,0.00,0.00}{#1}}
\newcommand{\DocumentationTok}[1]{\textcolor[rgb]{0.37,0.37,0.37}{\textit{#1}}}
\newcommand{\ErrorTok}[1]{\textcolor[rgb]{0.68,0.00,0.00}{#1}}
\newcommand{\ExtensionTok}[1]{\textcolor[rgb]{0.00,0.23,0.31}{#1}}
\newcommand{\FloatTok}[1]{\textcolor[rgb]{0.68,0.00,0.00}{#1}}
\newcommand{\FunctionTok}[1]{\textcolor[rgb]{0.28,0.35,0.67}{#1}}
\newcommand{\ImportTok}[1]{\textcolor[rgb]{0.00,0.46,0.62}{#1}}
\newcommand{\InformationTok}[1]{\textcolor[rgb]{0.37,0.37,0.37}{#1}}
\newcommand{\KeywordTok}[1]{\textcolor[rgb]{0.00,0.23,0.31}{\textbf{#1}}}
\newcommand{\NormalTok}[1]{\textcolor[rgb]{0.00,0.23,0.31}{#1}}
\newcommand{\OperatorTok}[1]{\textcolor[rgb]{0.37,0.37,0.37}{#1}}
\newcommand{\OtherTok}[1]{\textcolor[rgb]{0.00,0.23,0.31}{#1}}
\newcommand{\PreprocessorTok}[1]{\textcolor[rgb]{0.68,0.00,0.00}{#1}}
\newcommand{\RegionMarkerTok}[1]{\textcolor[rgb]{0.00,0.23,0.31}{#1}}
\newcommand{\SpecialCharTok}[1]{\textcolor[rgb]{0.37,0.37,0.37}{#1}}
\newcommand{\SpecialStringTok}[1]{\textcolor[rgb]{0.13,0.47,0.30}{#1}}
\newcommand{\StringTok}[1]{\textcolor[rgb]{0.13,0.47,0.30}{#1}}
\newcommand{\VariableTok}[1]{\textcolor[rgb]{0.07,0.07,0.07}{#1}}
\newcommand{\VerbatimStringTok}[1]{\textcolor[rgb]{0.13,0.47,0.30}{#1}}
\newcommand{\WarningTok}[1]{\textcolor[rgb]{0.37,0.37,0.37}{\textit{#1}}}

\usepackage{longtable,booktabs,array}
\usepackage{calc} % for calculating minipage widths
\usepackage{caption}
% Make caption package work with longtable
\makeatletter
\def\fnum@table{\tablename~\thetable}
\makeatother
\usepackage{graphicx}
\makeatletter
\newsavebox\pandoc@box
\newcommand*\pandocbounded[1]{% scales image to fit in text height/width
  \sbox\pandoc@box{#1}%
  \Gscale@div\@tempa{\textheight}{\dimexpr\ht\pandoc@box+\dp\pandoc@box\relax}%
  \Gscale@div\@tempb{\linewidth}{\wd\pandoc@box}%
  \ifdim\@tempb\p@<\@tempa\p@\let\@tempa\@tempb\fi% select the smaller of both
  \ifdim\@tempa\p@<\p@\scalebox{\@tempa}{\usebox\pandoc@box}%
  \else\usebox{\pandoc@box}%
  \fi%
}
% Set default figure placement to htbp
\def\fps@figure{htbp}
\makeatother



\ifLuaTeX
\usepackage[bidi=basic]{babel}
\else
\usepackage[bidi=default]{babel}
\fi
% get rid of language-specific shorthands (see #6817):
\let\LanguageShortHands\languageshorthands
\def\languageshorthands#1{}


\setlength{\emergencystretch}{3em} % prevent overfull lines

\providecommand{\tightlist}{%
  \setlength{\itemsep}{0pt}\setlength{\parskip}{0pt}}



 


\usepackage{booktabs}
\usepackage{longtable}
\usepackage{array}
\usepackage{multirow}
\usepackage{wrapfig}
\usepackage{float}
\usepackage{colortbl}
\usepackage{pdflscape}
\usepackage{tabu}
\usepackage{threeparttable}
\usepackage{threeparttablex}
\usepackage[normalem]{ulem}
\usepackage{makecell}
\usepackage{xcolor}
\makeatletter \expandafter\let\csname figure*\endcsname\figure \expandafter\let\csname endfigure*\endcsname\endfigure \expandafter\let\csname table*\endcsname\table \expandafter\let\csname endtable*\endcsname\endtable
\makeatletter
\@ifpackageloaded{tcolorbox}{}{\usepackage[skins,breakable]{tcolorbox}}
\@ifpackageloaded{fontawesome5}{}{\usepackage{fontawesome5}}
\definecolor{quarto-callout-color}{HTML}{909090}
\definecolor{quarto-callout-note-color}{HTML}{0758E5}
\definecolor{quarto-callout-important-color}{HTML}{CC1914}
\definecolor{quarto-callout-warning-color}{HTML}{EB9113}
\definecolor{quarto-callout-tip-color}{HTML}{00A047}
\definecolor{quarto-callout-caution-color}{HTML}{FC5300}
\definecolor{quarto-callout-color-frame}{HTML}{acacac}
\definecolor{quarto-callout-note-color-frame}{HTML}{4582ec}
\definecolor{quarto-callout-important-color-frame}{HTML}{d9534f}
\definecolor{quarto-callout-warning-color-frame}{HTML}{f0ad4e}
\definecolor{quarto-callout-tip-color-frame}{HTML}{02b875}
\definecolor{quarto-callout-caution-color-frame}{HTML}{fd7e14}
\makeatother
\makeatletter
\@ifpackageloaded{caption}{}{\usepackage{caption}}
\AtBeginDocument{%
\ifdefined\contentsname
  \renewcommand*\contentsname{Tabla de contenidos}
\else
  \newcommand\contentsname{Tabla de contenidos}
\fi
\ifdefined\listfigurename
  \renewcommand*\listfigurename{Listado de Figuras}
\else
  \newcommand\listfigurename{Listado de Figuras}
\fi
\ifdefined\listtablename
  \renewcommand*\listtablename{Listado de Tablas}
\else
  \newcommand\listtablename{Listado de Tablas}
\fi
\ifdefined\figurename
  \renewcommand*\figurename{Figura}
\else
  \newcommand\figurename{Figura}
\fi
\ifdefined\tablename
  \renewcommand*\tablename{Tabla}
\else
  \newcommand\tablename{Tabla}
\fi
}
\@ifpackageloaded{float}{}{\usepackage{float}}
\floatstyle{ruled}
\@ifundefined{c@chapter}{\newfloat{codelisting}{h}{lop}}{\newfloat{codelisting}{h}{lop}[chapter]}
\floatname{codelisting}{Listado}
\newcommand*\listoflistings{\listof{codelisting}{Listado de Listados}}
\makeatother
\makeatletter
\makeatother
\makeatletter
\@ifpackageloaded{caption}{}{\usepackage{caption}}
\@ifpackageloaded{subcaption}{}{\usepackage{subcaption}}
\makeatother

\usepackage{bookmark}
\IfFileExists{xurl.sty}{\usepackage{xurl}}{} % add URL line breaks if available
\urlstyle{same}
\hypersetup{
  pdftitle={Redes Neuronales Convolucionales \& Redes Recurrentes},
  pdfauthor={PhD. Pablo Eduardo Caicedo Rodríguez},
  pdflang={es},
  hidelinks,
  pdfcreator={LaTeX via pandoc}}


\title{Redes Neuronales Convolucionales \& Redes Recurrentes}
\subtitle{Regresión lineal múltiple, regresión logística y Redes
Neuronales}
\author{PhD. Pablo Eduardo Caicedo Rodríguez}
\date{2025-11-22}

\begin{document}
\frame{\titlepage}


\begin{frame}{Introduction}
\phantomsection\label{introduction}
\begin{tcolorbox}[enhanced jigsaw, title=\textcolor{quarto-callout-important-color}{\faExclamation}\hspace{0.5em}{Motivation}, colback=white, coltitle=black, bottomtitle=1mm, colbacktitle=quarto-callout-important-color!10!white, bottomrule=.15mm, left=2mm, breakable, toptitle=1mm, colframe=quarto-callout-important-color-frame, leftrule=.75mm, opacitybacktitle=0.6, toprule=.15mm, arc=.35mm, opacityback=0, rightrule=.15mm, titlerule=0mm]

Standard Feedforward Networks (MLPs) fail to scale for high-dimensional
data like images due to:

\begin{enumerate}
\tightlist
\item
  \textbf{Full Connectivity}: Exploding parameter count.
\item
  \textbf{Spatial Invariance}: Ignorance of local spatial topology.
\end{enumerate}

\end{tcolorbox}

\textbf{Convolutional Neural Networks (CNNs)} introduce:

\begin{itemize}
\tightlist
\item
  \textbf{Local Connectivity}: Neurons connect only to a local receptive
  field.
\item
  \textbf{Parameter Sharing}: Same weights (filters) applied across the
  input.
\item
  \textbf{Equivariance}: Translation of input results in translation of
  output.
\end{itemize}
\end{frame}

\begin{frame}{The Convolution Operation}
\phantomsection\label{the-convolution-operation}
\begin{columns}[T]
\begin{column}{0.45\linewidth}
In the context of CNNs, the operation is technically a
\textbf{cross-correlation}, but conventionally termed convolution.

Given an input image \(I\) and a kernel (filter) \(K\), the feature map
\(S\) is defined as:
\end{column}

\begin{column}{0.45\linewidth}
\pandocbounded{\includegraphics[keepaspectratio]{../../recursos/imagenes/GIFs/heatmap_moving_mask.gif}}
\end{column}
\end{columns}

\[S(i, j) = (I * K)(i, j) = \sum_{m} \sum_{n} I(i+m, j+n) K(m, n)\]

Where: * \((i, j)\) are the pixel coordinates. * \((m, n)\) are the
kernel offsets.
\end{frame}

\begin{frame}{Hyperparameters}
\phantomsection\label{hyperparameters}
The spatial dimensions of the output feature map depend on:

\begin{enumerate}
\tightlist
\item
  \textbf{Filter Size (\(F\))}: Receptive field dimensions (e.g.,
  \(3 \times 3\)).
\item
  \textbf{Stride (\(S\))}: Step size of the filter convolution.
\item
  \textbf{Padding (\(P\))}: Zero-padding around the border to preserve
  dimensions.
\end{enumerate}

\textbf{Output Dimension Formula:} Given input size
\(W_{in} \times H_{in}\):

\[W_{out} = \frac{W_{in} - F + 2P}{S} + 1\]
\end{frame}

\begin{frame}{Pooling Layers}
\phantomsection\label{pooling-layers}
Pooling provides \textbf{invariance to small translations} and reduces
dimensionality (downsampling).

\begin{block}{Max Pooling}
\phantomsection\label{max-pooling}
Selects the maximum activation in the receptive field:
\[y_{i,j,k} = \max_{(p,q) \in \mathcal{R}_{i,j}} x_{p,q,k}\]
\end{block}

\begin{block}{Average Pooling}
\phantomsection\label{average-pooling}
Calculates the arithmetic mean. Generally, Max Pooling performs better
for identifying dominant features (edges, textures).
\end{block}
\end{frame}

\begin{frame}{Activation Functions}
\phantomsection\label{activation-functions}
Linear convolution is insufficient for approximating non-linear
functions.

\textbf{ReLU (Rectified Linear Unit):} \[f(x) = \max(0, x)\]

\begin{itemize}
\tightlist
\item
  \textbf{Sparsity}: Activations \(< 0\) are zeroed out.
\item
  \textbf{Gradient Propagation}: Mitigates vanishing gradient problem
  compared to Sigmoid/Tanh.
\end{itemize}

Variants: \emph{Leaky ReLU}, \emph{ELU}, \emph{GELU} (Gaussian Error
Linear Unit).
\end{frame}

\begin{frame}{Architecture Overview}
\phantomsection\label{architecture-overview}
A typical CNN architecture follows a hierarchical pattern:

\begin{enumerate}
\tightlist
\item
  \textbf{Feature Extraction Block}:

  \begin{itemize}
  \tightlist
  \item
    {[}Conv \(\rightarrow\) ReLU \(\rightarrow\) Pooling{]} \(\times N\)
  \end{itemize}
\item
  \textbf{Classification Head}:

  \begin{itemize}
  \tightlist
  \item
    Flattening
  \item
    Fully Connected Layers (Dense)
  \item
    Softmax (for multi-class classification)
  \end{itemize}
\end{enumerate}

\[P(y=j | \mathbf{x}) = \frac{e^{\mathbf{w}_j^T \mathbf{h} + b_j}}{\sum_{k=1}^K e^{\mathbf{w}_k^T \mathbf{h} + b_k}}\]
\end{frame}

\begin{frame}{Backpropagation in CNNs}
\phantomsection\label{backpropagation-in-cnns}
Training requires computing gradients w.r.t weights \(W\) using the
Chain Rule.

For a convolution layer \(l\):
\[\frac{\partial L}{\partial W^{(l)}} = \frac{\partial L}{\partial \text{out}^{(l)}} * \text{in}^{(l)}\]

Where the gradient is computed via convolution between the incoming
error signal and the input activations from the previous layer.
\end{frame}

\begin{frame}[fragile]{Implementation: PyTorch Snippet}
\phantomsection\label{implementation-pytorch-snippet}
\begin{Shaded}
\begin{Highlighting}[]
\ImportTok{import}\NormalTok{ torch}
\ImportTok{import}\NormalTok{ torch.nn }\ImportTok{as}\NormalTok{ nn}

\KeywordTok{class}\NormalTok{ SimpleCNN(nn.Module):}
    \KeywordTok{def} \FunctionTok{\_\_init\_\_}\NormalTok{(}\VariableTok{self}\NormalTok{):}
        \BuiltInTok{super}\NormalTok{(SimpleCNN, }\VariableTok{self}\NormalTok{).}\FunctionTok{\_\_init\_\_}\NormalTok{()}
        \CommentTok{\# Feature Extraction}
        \VariableTok{self}\NormalTok{.features }\OperatorTok{=}\NormalTok{ nn.Sequential(}
\NormalTok{            nn.Conv2d(in\_channels}\OperatorTok{=}\DecValTok{1}\NormalTok{, out\_channels}\OperatorTok{=}\DecValTok{32}\NormalTok{, kernel\_size}\OperatorTok{=}\DecValTok{3}\NormalTok{, padding}\OperatorTok{=}\DecValTok{1}\NormalTok{),}
\NormalTok{            nn.ReLU(),}
\NormalTok{            nn.MaxPool2d(kernel\_size}\OperatorTok{=}\DecValTok{2}\NormalTok{, stride}\OperatorTok{=}\DecValTok{2}\NormalTok{),}
\NormalTok{            nn.Conv2d(}\DecValTok{32}\NormalTok{, }\DecValTok{64}\NormalTok{, kernel\_size}\OperatorTok{=}\DecValTok{3}\NormalTok{, padding}\OperatorTok{=}\DecValTok{1}\NormalTok{),}
\NormalTok{            nn.ReLU(),}
\NormalTok{            nn.MaxPool2d(}\DecValTok{2}\NormalTok{, }\DecValTok{2}\NormalTok{)}
\NormalTok{        )}
        \CommentTok{\# Classifier}
        \VariableTok{self}\NormalTok{.classifier }\OperatorTok{=}\NormalTok{ nn.Sequential(}
\NormalTok{            nn.Flatten(),}
\NormalTok{            nn.Linear(}\DecValTok{64} \OperatorTok{*} \DecValTok{7} \OperatorTok{*} \DecValTok{7}\NormalTok{, }\DecValTok{128}\NormalTok{), }\CommentTok{\# Assuming 28x28 input}
\NormalTok{            nn.ReLU(),}
\NormalTok{            nn.Linear(}\DecValTok{128}\NormalTok{, }\DecValTok{10}\NormalTok{)}
\NormalTok{        )}

    \KeywordTok{def}\NormalTok{ forward(}\VariableTok{self}\NormalTok{, x):}
\NormalTok{        x }\OperatorTok{=} \VariableTok{self}\NormalTok{.features(x)}
\NormalTok{        x }\OperatorTok{=} \VariableTok{self}\NormalTok{.classifier(x)}
        \ControlFlowTok{return}\NormalTok{ x}
\end{Highlighting}
\end{Shaded}
\end{frame}

\section{Recurrent Neuronal Network}\label{recurrent-neuronal-network}




\end{document}
